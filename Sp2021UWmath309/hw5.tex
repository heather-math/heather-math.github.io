\documentclass[11pt]{article}
\usepackage{amsmath, amscd, amssymb, amsthm}
% \usepackage{diagrams}
\usepackage{color}
\usepackage{graphicx, psfrag, tikz}
\usepackage[all]{xy}
\usepackage[margin=1.1in]{geometry}

\usepackage{enumitem,kantlipsum}
\usetikzlibrary{arrows.meta}
\usetikzlibrary{calc,patterns,angles,quotes}
%\renewcommand{\baselinestretch}{1.17}
\usepackage{chemformula}

\newtheorem{theorem}{Theorem}[section]
\newtheorem{lemma}[theorem]{Lemma} 
\newtheorem{proposition}[theorem]{Proposition}
\newtheorem{corollary}[theorem]{Corollary}
\theoremstyle{definition} 
\newtheorem{definition}[theorem]{Definition}
\newtheorem{conjecture}[theorem]{Conjecture}
\newtheorem{remark}[theorem]{Remark}
\newtheorem{example}[theorem]{Example}


\begin{document}
\begin{center}
\textbf{Math 309 Homework 5}\\
(4 problems)
\end{center}
\vspace{0.15in}


\begin{enumerate}[leftmargin=*]

\item (Continuation of HW4 \#6) Consider the system $x'=Ax$ from HW4 \#6, i.e.\[
x'=\left[
\begin{array}{cc}
1 & 2\\
-5 & -1
\end{array}\right]x.
\]

\begin{itemize}
\item[(a)]  In HW4 \#6a, you found the general solution $x=\left[
\begin{array}{c}
x_1\\
x_2
\end{array}\right]$ of the above system in terms of real valued functions.  Verify that 
\[
\left(\frac{x_1}{2}\right)^2+\left(\frac{2x_2+x_1}{6}\right)^2=c
\]
for some positive constant $c$.  This are ellipses with tilted axes.\\

\item[(b)] Draw a few trajectories in the $x_1$-$x_2$ plane.  You need to draw the ellipses precisely, i.e. you will need to figure out precisely the axes of the ellipses. You can follow the hint below which describes two different methods for doing this.  You just need to choose one (method 1 is more recommended).  \\




\textbf{Method 1}: Write the equation you verified in part (a) in the form
\[
x^TQ x =\left[\begin{array}{cc} x_1 & x_2\end{array}\right]Q \left[\begin{array}{c} x_1 \\ x_2\end{array}\right]=c,
\]
where $Q$ is a symmetric matrix.  After you find $Q$,  try to diagonalize it by writing $Q$ as $Q=PDP^{-1}$ for some matrix $P$ and a diagonal matrix $D$ (feel free to use a calculator for this).  Let $y=P^{-1} x$.  Then in the $y$ coordinate, the equation of the  ellipse is 
\[
y^TDy=c.  
\]
Convince yourself that this is a straight ellipse with axis being the $y_1$ and $y_2$ axes in the $y_1$-$y_2$ plane.  Convince yourself  that the axes of the ellipse in the $x_1$-$x_2$ plane are generated by columns of $P$ (you can think of $P$ as a linear transformation that takes the basis vectors to the columns of $P$).   Note that the eigenvectors of $Q$ are automatically orthogonal because $Q$ is symmetric.\\

\textbf{Edit:}  sorry there's a mistake above but don't worry it won't  change your answer.  In the above, it should be $y^T\widetilde{D}y=c$, for some diagonal matrix $\widetilde{D}$, i.e. this $\widetilde D$ is not necessarily the same $D$ as the diagonal matrix in $Q=PDP^{-1}$.\\


\textbf{Method 2}: If an axis of an ellipse goes thru a point $\left[
\begin{array}{c}
x_1\\
x_2
\end{array}\right]$ on the ellipse, then this axis is perpendicular to the tangent vector $\left[
\begin{array}{c}
x_1'\\
x_2'
\end{array}\right]$ at this point.  Then find out for which $(x_1, x_2)$ is    $\left[
\begin{array}{c}
x_1\\
x_2
\end{array}\right]$ perpendicular to $\left[
\begin{array}{c}
x_1'\\
x_2'
\end{array}\right]$, i.e. their dot product is zero.  This will provide you with one axis.  The other axis of the ellipse is perpendicular to this one.  \\\end{itemize}


\item
(Pendulum, a continuation of HW1 \#8, HW3 \#5, HW4 \#3)  (1pt) This problem will be graded on effort.  I will briefly discuss this problem in L14, which should help you understand what I'm asking you to do for this problem.  This problem looks long, but all it asks you to do is to draw a picture at the end.

Recall that in HW1 \#8, we discussed that the equation of motion for a frictionless pendulum of bob mass $m$ and rod length $L$ is
\[
\theta''(t)=-\frac{g}{L}\sin \theta.
\]
Note that if  $\theta(t)$ is a constant solution, then $\theta''(t)=0$, so $\sin \theta=0$.  From this we can see that the set of constant solutions are 
\[
\{\theta(t)=n\pi\mid n= 0, \pm 1, \pm 2, \pm 3, \ldots\}
\]
Near each of these constant values, $n\pi$, one can consider the Taylor expansion of $\sin \theta$ at that point and see these approximations
\[
\sin\theta = \sin(n\pi)+\sin'(n\pi)(\theta-n\pi)+\cdots \approx \begin{cases} 
+(\theta-n\pi)& \text{ if } \theta\approx n\pi,\  n \text { even},\\
- (\theta- n\pi) & \text{ if } \theta \approx n\pi, \ n \text{ odd}.
\end{cases}
\]
Note that we did the case when $\theta\approx 0\pi$ in HW1 \#8 and HW4 \#3, and we did the case when $\theta\approx \pi$ in HW3 \#5.  

Let 
\[
\tilde \theta= \theta -n\pi,
\]
and let $\omega=\frac{d\theta}{dt}$.
Then we have the following two cases.  

\begin{itemize}
\item \textbf{If $n$ is even}, then $\sin \theta\approx \tilde \theta$ and the approximate equation of motion is 
\[
\tilde \theta'' = -\frac{g}{L} \theta,
\]
i.e. 
\[
\left[ \begin{array}{c}\tilde  \theta' \\ \omega' \end{array}\right]= \left[ \begin{array}{cc} 0 & 1 \\ -\frac{g}{L} &0 \end{array}\right]\left[ \begin{array}{c}\tilde  \theta \\ \omega \end{array}\right].
\]
You solved this equation and drew some trajectories in the $\tilde \theta$-$\omega$ plane when $\frac{g}{L}=1$ in HW4 \#3.

\item  \textbf{If $n$ is odd}, then $\sin \theta\approx - \tilde \theta$ and the approximate equation of motion is 
\[
\tilde \theta'' = \frac{g}{L} \theta,
\]
i.e. 
\[
\left[ \begin{array}{c}\tilde  \theta' \\ \omega' \end{array}\right]= \left[ \begin{array}{cc} 0 & 1 \\ \frac{g}{L} &0 \end{array}\right]\left[ \begin{array}{c}\tilde  \theta \\ \omega \end{array}\right].
\]
You solved this equation and drew some trajectories in the $\tilde \theta$-$\omega$ plane when $\frac{g}{L}=1$ in HW3 \#5.
\end{itemize}
Now on a single $\theta$-$\omega$ plane.  The constant solutions are $(\theta,\omega)=(n\pi, 0)$.  Near each such point, you have a local picture of the trajectories that looks the same as what you drew in HW3 and HW4, for $n$ odd and even, respectively.  Try to connect those trajectories to draw a global picture of the trajectories in the entire $\theta$-$\omega$ plane.  Then try to follow some of these trajectories along and think about how the pendulum behaves and you along the trajectories in the $\theta$-$\omega$ plane.


\item
\begin{itemize}
\item[(a)] Find the general solution to the given system 
\[
x'=\left[
\begin{array}{cc}
1 & -4\\
4 & -7 
\end{array}\right]x
\]
\item[(b)] Sketch a few trajectories in the $x_1$-$x_2$ plane and describe the behavior of the solution as $t\to \infty$.
\item[(c)]  Find the solution with the initial value
\[
x(0)=\left[\begin{array}{c}
3 \\
2
\end{array}\right].
\]
\item[(d)] Draw the trajectory of the solution to part (c) in the $x_1$-$x_2$ plane.  \\
\end{itemize}


\item 
\begin{itemize}
\item[(a)] Find the general solution to the given system 
\[
x'=\left[
\begin{array}{cc}
3 & 9\\
-1 & -3 
\end{array}\right] x
\]
\item[(b)] Sketch a few trajectories in the $x_1$-$x_2$ plane and describe the behavior of the solution as $t\to \infty$.
\item[(c)] Find the solution with the initial value
\[
x(0)=\left[\begin{array}{c}
2 \\
4
\end{array}\right].
\]
\item[(d)] Draw the trajectory of the solution to part (c) in the $x_1$-$x_2$ plane.  
\end{itemize}
\end{enumerate}





\end{document}
